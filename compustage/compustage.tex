% ============================ compustage.tex ========================
% 3456789012345678901234567890123456789012345678901234567890123456789012
% Version of 04-Jan-2013
% Assumes a directory structure
% ../work/
% ../work/global/...
% ../work/proj/name/dat/...
% ../work/proj/name/scripts/...
% ../work/proj/name/TeX/...
% ../work/proj/name/Sweave/...
% ../work/proj/name/R/...
% Works well for 800x600 images. Might need to change width for
% other aspect ratios...

\documentclass[12pt]{article}
\usepackage[margin=0.75in, letterpaper]{geometry} 

% define some shortcuts for this document
% in the spirit of DRY
\newcommand{\theAuthor}{FEI and J. Minter}
\newcommand{\theTitle}{FEI CM20UT CompuStage Instructions}
\newcommand{\theSubject}{My subject}
\newcommand{\theDate}{\today}
\newcommand{\theKeywords}{some, key, words}

% favorite packages
\usepackage{graphicx}
\usepackage{subfig}
\usepackage{textcomp} % Greek in text mode - \textmu
\usepackage{hyperref} % urls - \href{url}{text}

\hypersetup
{
    pdfauthor={\theAuthor},
    pdfsubject={\theSubject},
    pdftitle={\theTitle},
    pdfkeywords={\theKeywords}
}

\usepackage{verbatim} 
%\usepackage[parfill]{parskip} % begin paragraphs with empty line
                             
\usepackage{color}
\usepackage{microtype}

% define colors
\definecolor{kodak-red}{RGB}{204,0,0}
\definecolor{kodak-yellow}{RGB}{255,204,0}

% define degree symbol
\newcommand{\degree}{\ensuremath{^\circ}}

% define commands for subscripts & superscripts to use in text mode
\newcommand{\superscript}[1]{\ensuremath{^{\textrm{#1}}}}
\newcommand{\subscript}[1]{\ensuremath{_{\textrm{#1}}}}


% The preamble for this document...
% In case I want a TOC
\title{\theTitle}
\author{\theAuthor}
\date{\theDate}

\begin{document}
\bibliographystyle{IEEEtran}	 % (uses file "IEEEtran.bst")

% un-comment to make a standard title
% \maketitle


% I prefer more control over a title, and so use this
\begin{center}
\textbf{\LARGE{\theTitle}}
\end{center}

\begin{center}
\theAuthor
\footnote{These updated instructions for the FEI CM20UT CompuStage are
adapted from the Technai F20/30 column manual.}
\end{center}

\vspace{0.125in}
\hrule
\vspace{0.125in}

The CompuStage is a motor-driven goniometer that provides
computer-controlled movement of the specimen on five axes (X, Y, Z,
$\alpha$, $\beta$). The CompuStage consists of the following elements:
\begin{itemize}
\item Hardware (the goniometer), including motor drives and
position-measuring system.
\item Control electronics, including a dedicated microprocessor.
\item CompuStage server software which provides the link between the
user interface (software and hardware controls like the track ball
for X-Y motion) and the CompuStage microprocessor.
\item Hardware controls, consisting of a track ball for X-Y motion and
pressure-sensitive up-down switches for Z, $\alpha$, $\beta$.
\item User interface software with translates operator input into
goniometer actions. The definition of the physical position of the stage
axes in the microscope column is covered in a separate section.
\end{itemize}

\section{Specimen holders}
The CompuStage can be equipped with a variety of holders, which are
inserted or removed via an airlock. Specimen holders are inserted into
ultra-high vacuum and must be kept clean. Special handling
instructions for specimen holders should be adhered to.

\section{Eucentric height}
The a tilt of the CompuStage is constructed in such a way that it is
possible to tilt around it without having large apparent movements of
the point of interest on the specimen. This is called eucentric tilting
and is achieved by bringing the point of interest to the same height
(with the Z axis) as the a tilt axis itself: the eucentric height. The
eucentric height is important because it not only provides an easy way
of tilting without having to correct specimen position continuously,
but it also defines the reference point inside the microscope for all
alignments, magnification, camera lengths, and so on. In general one
should work at the eucentric height (the only reason for deviating
could be that at very high $\beta$ tilts and specimen positions away
from the center, the range of the Z axis may not be sufficient to
bring the specimen to the eucentric height).

\subsection{To set the specimen to the eucentric height}

\subsubsection{Method 1 Eucentric Focus Preset}

Once the microscope has been aligned (as it normally should be), the
eucentric focus preset (obtained by pressing the Eucentric Focus
button) sets the objective-lens setting to the correct value for a
specimen that would be exactly at the eucentric height. If the specimen
is not at the eucentric height, it will appear out of focus. If you now
bring the specimen into focus by changing the Z height, not the
focus, it will go to the eucentric height. It may help to switch on the
wobbler, since the apparent displacement between the two wobbler images
makes it easy to see whether the height is changed in the right
direction (the displacement between the images becomes smaller).

\subsubsection{Method 2 The Alpha Wobbler}

Since the image displacement on tilting is minimised at the eucentric
height, the eucentric height can be set by minimising the displacement
while the stage is tilting. For this purpose the CompuStage has the
Alpha Wobbler function. When this function is activated, the CompuStage
is tilted continuously between two preset tilts (typically -15 and
+15\degree ). Change the Z height to make the displacement smaller. When
the displacement is minimised the specimen is at the eucentric height.


\section{Safety features}
For safety the CompuStage is equipped with a number of special features
like the MaxiTilt system which allows maximum use of the available space
between the pole piece for tilting while guarding against damage of
holder, pole pieces and other objects in the pole-piece gap like the
objective aperture holder.

Another feature is the sEntry system which guards against insertion of
holders that are incompatible (too big) for the objective-lens
pole-piece configuration of the microscope.

\subsection{MaxiTilt system}

Because tilting takes place around the eucentric point, the motion of
the specimen holder may require a lot of space (see picture below).
In the restricted space available between the pole pieces of some
objective lens types, there may not be enough room to tilt very far.
But the pictures above demonstrate that the available tilt range is
very much dependent on the position of the stage (mostly the Y
and Z axes). The MaxiTilt system of the CompuStage provides a flexible
way of keeping the tilt within a safe range, while at the same time
maximising the tilt available. The MaxiTilt senses what the maximum
tilt is for the current stage position. If this range is exceeded
(a situation called a pole hit), the CompuStage will move a little bit
back on the axis that was changed last.

If there is no axis identifiable, the $\beta$ or $\alpha$ tilt will move
back. The microscope will display an information message that a pole
hit has been detected.


\subsection{sEntry system}

The CompuStage is equipped with a SafeEntry (or sEntry) system that
prevents holders from being inserted into microscopes where the
objective-lens pole-piece configuration is not compatible with the
particular holder (like thick holders into objective lenses with a
gap that is too narrow for the holder to fit).

The sEntry system consists of a key (a pin varying in shape and
diameter) on the holder defining the holder dimensions and a lock on
the CompuStage defining the pole-piece configuration. The length of
the sEntry key is such that if an incompatible holder is inserted,
the key (blocked by the lock) prevents the holder from going between
the pole pieces.

Although older specimen holders designed fo the manual goniometer will
fit in the CompuStage (provided their O-ring is exchanged for a
-- thicker -- CompuStage O-ring), these holders are not equipped with a
sEntry key and may therefore be unsafe with objective pole-piece
configurations with narrow gaps such as the U-TWIN.

\section{Red CompuStage light}

The red CompuStage light has a more generic function than previously
used on manual goniometers. It simply means that no airlock actions
(insertion or removal) should be executed. This will happen of
course while the airlock is being pumped when a holder is inserted into
the microscope. It can also mean
that it is unsafe to extract the holder under the current conditions
(when the $\beta$ tilt is more than 5 degrees
or during movement the red light will also be on). In that case, the
unsafe situation must be rectified before the holder can be extracted
(reset $\beta$ tilt to zero or wait until movement is finished).


\section{Homing}

Before the CompuStage is ready for use (when the microscope has been
switched off altogether or after the CompuStage has been disabled), it
must be homed, a procedure in which it finds the zero positions.

During homing the CompuStage will move each of the four fixed axes (X,
Y, Z and $\alpha$; the $\beta$ tilt axis is homed separately whenever
a double-tilt holder is inserted) to one extreme. Because of the high
tilt applied, this procedure must be executed without a specimen holder.
In order to make sure that there is no specimen holder present, the
microscope has to ask the operator to identify the specimen holder
(which in this case should be 'No specimen holder', otherwise the
homing cannot proceed).

If no user interface is active, there is no way for communicating with
the operator. Consequently, the homing procedure after a start-up will
only proceed once the user interface has started. Even if the user
interface does not show a message to select a specimen holder after a
restart of the microscope, but the CompuStage does not move (easily
checked by moving the right-hand track ball a bit), it is very likely
that the CompuStage has not been homed. Select the Stage Settings
Control Panel and press the Enabled button. The homing procedure will
start.

\section{Parking position}

Occasionally it is necessary to get the specimen and/or specimen holder
out of the way (for example, when no electron beam is visible initially
or in some alignment steps if the specimen blocks the beam).
Although it is possible to retract the holder fully and turn it slightly
(the initial part of removing the specimen holder), it is better not to
do this, because it can lead to a potential leak. The preferred method
to `remove' the specimen holder is to retract\footnote{Always retract
the holder slowly to allow the O-ring to keep the seal.} it about one
or two centimeters and stick something elongated (a pen with a diameter
of a centimeter or so will do nicely\footnote{Manual-goniometer specimen
holders with the sapphire at the end have a longer tip than CompuStage
holders and must be retracted at least three centimeters before they
are clear from the field of view.}) between the inside cap of the
holder and the CompuStage. The 'elongated' item will prevent the holder
from moving back in and the distance is sufficient to remove any
obstruction from the field of view.

\section{Specimen-holder handling}

\subsection{Specimen-holder selection}

There are two ways for informing the microscope which specimen holder
is being used:
\begin{itemize}
\item \textbf{Not automatic:} Each time a holder is inserted, the
holder must be selected from the list that pops up in the message area
of the Tecnai user interface.

\item \textbf{Default:} Only for single-tilt holders, the last-used
setting is selected automatically.
\end{itemize}

The CompuStage is equipped with the sEntry system. Associated with this
system is an up-down switch on the outer panel of the CompuStage. The
top part of the switch depends on the type of objective lens
(with smaller lens gaps, the hole in the top is made smaller so only
specific types of holders will fit).

The shield of the CompuStage showing the position of the
holder-selection switch (gray). The switch itself is controlled by a
notch in the lower rectangle. The actual shape of the top of the switch
varies and on some microscope will block the sEntry key of the specimen
holder from being inserted.

\begin{itemize}
\item  In the up position (left-hand picture above), the Default holder
selection is enabled. In this case the microscope will automatically
select a single-tilt holder. In case there are more single-tilt holders
present on the system (e.g. normal single- tilt and cryo holder), the
holder selected is the last one chosen. If there are more single-tilt
holders present, then for the first insertion, move the switch
down, insert and identify the holder you are using, then move the
switch up again. Thereafter the microscope will automatically select
the same holder again.

\item  In the down position, the holder used must be identified by the
user. This must always be done for double-tilt holders (the reason this
is necessary has to do with the $\beta$ tilt cable which must be
connected before the holder can be inserted into the microscope), so do
not use the switch up position with double-tilt holders.
\end{itemize}

\subsection{Handling instructions}

Specimen holders are a bridge between the air pressure outside the
column and the ultra-high vacuum inside. Their cleanliness is an
important factor in keeping contamination down (specimen holders are the
second-most important source of the contamination - specimens themselves
are the primary source nowadays). Caution should therefore applied to
handling specimen holders. The following instructions should be
adhered to:

\begin{itemize}
\item  Always use clean nylon or similar gloves when handling
specimen-holder parts that enter the vacuum (that is, between the tip
of the holder and the sealing O-ring).

\item  Clean the tip of the holder only with special cleaning fluids or
with a fresh piece of window-cleaning (chamois) leather.

\item  Specimen, spacing washers and clamping devices should be
manipulated only using pointed tweezers or the tools provided. Tools
like the hex-ring tool or the needle for levering the single-tilt
holder clamp should never be touched by hand on the wrong side (in
the case of the hex-ring tool the use of gloves is advised because it
is easy to pick it up at the wrong end). Clean the washers, tools
and tweezers on a regular basis.

\item  The O-ring on the specimen-holder rod should be checked for
possible dirt or excessive quantities of grease although it should not
be completely dry. A very light coating of Fomblin grease (supplied with
the microscope) is advised. Take care not apply grease to the conical
part of the holder (the part between the thicker and thinner sections
of the rod). The conical part is the area where the holder is
`seated' in the CompuStage. Any grease there will very likely result
in drift in excess of specimen.

\item  When a specimen holder is not in use, insert it in the protective
holder cover supplied or reinsert it into the microscope. The latter
keeps the holder thermally equilibrated with the microscope and
CompuStage, thereby reducing drift after holder insertion (especially
if there is a considerable
temperature difference between the room and the microscope column).
\end{itemize}

\subsection{Inserting a specimen into the single-tilt holder}

(Instructions for mounting specimens in other holders are covered in
the manual accompanying these holders.)
Note: For easy removal of specimens from the holder, a notch is provided
on the holder (see the images below). One tip of a pair of tweezers
can be inserted in this notch underneath the specimen (once the clamp
has been lifted), making it easy to take the specimen out of the holder.

\begin{itemize}
\item  If necessary, remove the cap at the end of the tube of the
specimen-holder cover.
\item  Check that the tip of the holder and the clamping device are
clean and dry.
\item  Keep one hand against the cap of the holder, making sure it
cannot move out of the cover tube.
\item  Fit the tool (stored in one of the holes in the supports of
the cover tube) into the hole in front of the clamp. Then lift the
clamp to its fullest extent.
\item  Place the specimen in the (roughly) circular recess of the
specimen-holder tip. If the specimen is on a re-insertable grid, place
the `ear' of the grid in the 'ear' recess.
\item  Carefully lower the clamp with the tool onto the specimen. Make
sure the specimen remains correctly in position. \textbf{Caution:}
The specimen-securing clamp must be lowered carefully, otherwise the
specimen and/or clamp can be damaged.
\item  Retract the holder slightly in the cover and turn it upside down.
Tap the cap at the end a few times. Turn the holder back and check that
the specimen has not moved (movement is a sign that it isn't clamped
properly). Note: Never mount magnetic specimens (disks) in the
single-tilt holder. The clamp is normally not strong enough to prevent
the specimen from flying out due to the objective-lens magnetic field
and sticking to the objective-lens pole pieces.
\end{itemize}

\subsection{Inserting a specimen holder into the microscope}

Caution: The following instructions apply to all specimen holders and
must be followed completely or damage to airlock, specimen holder or
specimen stage may result.\footnote{Read the specimen-holder handling
instructions before proceeding.}

The FEG microscopes are equipped with a
turbo-molecular pump: The turbo-molecular pump (which should not remain
running under normal microscopy because of the vibrations it causes)
takes several minutes (2-3) to reach operational speed. To speed up
specimen exchange, it is advised to switch the pump on (use the toolbar
button) before extracting the holder. By the time the specimen has
been exchanged the turbo-molecular pump will be near or at its operation
speed and pumping on the airlock will begin (almost) immediately. Switch
the pump off again after the holder has been inserted fully into the
microscope.

If the turbo-molecular pump is running on when the specimen holder is
inserted into the airlock, the pump will first spin up and only after a
few minutes start pumping on the airlock. In this situation (holder
triggers the pump), the pump will be switched off automatically after
pumping on the airlock has finished.

The specimen airlock of the CompuStage and the specimen holder consist
of fine, high-quality mechanics. If considerable force is needed for
any manual actions on the holder or CompuStage, it is a sign of
something being wrong. It should never be necessary to exert strong
force and doing so may well result in damage to specimen holder or
CompuStage.

It is not necessary to switch off high tension or filament during
specimen-holder insertion, since the gun is separated from the column
by the gun valve (V7) which is closed during specimen insertion
(even if the user doesn't close the column valves, the microscope will
automatically do so when it detects that a specimen holder has been
inserted; it is good practice, however, to close the column valves
before extracting the specimen holder from the microscope so they will
still be closed when a holder is inserted again). It is advised,
however, to keep the column valves closed after insertion of the holder
(typically for a few minutes) while the specimen-area vacuum
recovers -- if only to reduce contamination of the specimen. Using the
cold trap is advised to trap water vapour quickly (ion-getter pumps
have difficulty in pumping water).

Always carry out the complete insertion procedure. If the specimen is
left in a retracted position, vacuum leakage can occur with consequent
contamination (or, if left for a long period, loss of vacuum in
the airlock and a crash of the column if the specimen holder is then
inserted). If you do not want to insert the specimen holder after all
(or if the airlock isn't pumped properly, for example because of a
leaking O-ring), leave the airlock cycle to finish pumping (the red
LED on the CompuStage goes off) before removing the holder from the
airlock. Do not extract the holder while the airlock is being pumped.
The specimen holder is introduced through a pre-pumped airlock which
ensures that air, introduced with the holder, is pumped away before the
airlock is opened to the microscope column.

\subsection{Procedure}
\begin{itemize}
\item Hold the specimen holder with the airlock trigger pin parallel
to the small slit in the CompuStage front plate (at roughly four
o'clock). Carefully insert the end of the specimen holder into the
airlock cylinder and slide the holder in until a stop is reached. At
this point the prepumping of the airlock will start as indicated by the
red CompuStage light which will be illuminated. If the light does not
come on, the trigger has not been positioned correctly. Slowly turn
the holder slightly to the left and the right until it will go in a
bit further (the airlock trigger pin now falls properly into its groove).

\item The Tecnai user interface will display a message asking for
identification of the specimen holder. Select the type of holder from
the list and press the Enter button.

\end{itemize}

\section{Stage movement}

The specimen stage movement is controlled by a track ball (normally the
right-hand one, but the assignment can be changed by the user) and/or
by Multifunction knob. The movement has two modes of operation:

\begin{itemize}
\item The track ball (discontinuous movement) mode.
\item The `joy stick' (continuous movement) mode.
\end{itemize}

In the track ball mode (the default) the stage will move whenever the
track ball has been moved, in the direction and with a displacement
related to the direction and displacement of the track ball. As soon as
the track ball stops moving, the stage will stop moving as well. For
normal operation at moderate to high magnifications this is the
preferred mode. However, for searching at low magnifications the
track ball mode requires that the user keeps moving the track ball
continuously.

In the 'joy stick' mode the stage will move in the direction indicated
by the movement of the track ball and with a speed related to the
displacement of the track ball, and the stage will keep moving in the
indicated direction without requiring further input through the track
ball. The direction and speed of movement can be influenced by further
control of the track ball. To stop the stage movement in the `joy stick'
mode, press one of the track ball buttons.

Switching between the two modes is achieved by pressing the two track
ball buttons simultaneously.

\subsection{Speed control}

The speed of movement of the stage is related to three parameters:

\begin{itemize}
\item The speed value setting (1 to 9) as defined by pressing the
left- (speed down) and right-hand (speed up) track ball buttons.
\item The current magnification.\footnote{One exception is at the
lowermost speed setting. In this case the CompuStage will make its
smallest steps, independent of magnification. At low magnifications
this may mean that the stage doesn't seem to move at all. If the latter
is the case, click once on the right-hand track-ball button to switch
the speed one step up.}\footnote{ At very low magnifications the speed
control may appear to function no longer (speed up doesn't increase the
speed of the CompuStage). This means that the CompuStage has reached its
maximum speed and can go no faster.}
\item The displacement of the track ball.
\end{itemize}

\subsection{Specimen stage movement by Multifunction knob}

The specimen stage movement can also be assigned to the Multifunction
knobs. In this case the Multifunction knobs duplicate the track ball
directions (that is, Multifunction X = track ball X). The knobs
are thus not connected directly to a stage axis (technically this is
not possible within the software architecture). For most microscopes
this means that the X axis (the direction of the a tilt axis) is
approximately connected to the Multifunction Y knob.

\subsection{Stage Axes}

The CompuStage is a goniometer with five axes. Three of these are
orthogonal translation movements X, Y and Z. The other two are mutually
perpendicular rotation movements, a and b. The a tilt is parallel
to the X axis and b tilt parallel to the Y axis.

The orientation in space of these axes can only be described at a tilt
at 0\degree\ (since the other axes are mounted on top of the a tilt,
they will change their orientation with it). At a tilt 0\degree\ the X
and Y axes are horizontal and the Z axis vertical. The X axis runs
along the rod of the specimen holder. If you look down
on the microscope and define the front of the microscope as south, the
X axis thus runs NW-SE, with SE in the + direction and the Y axis NE-SW
with NE as the + direction. Up is the + direction of the Z axis.

The X axis describes a truly linear motion, the Y and Z are in fact
circular motions but with such a wide radius that the motion remains
close to linear. The Y and Z motions cause the specimen holder rod to
pivot around the conical face where it narrows down, just beyond the
O-ring.

Because of this pivoting motion, the end of the holder on the outside
of the CompuStage moves in the opposite direction! Below is a schematic
3-D view of the specimen holder tip and the orientation of the stage
axes.

Note: The notation differs from that on the earlier CompuStage of the
CM microscopes where the sign of the X and Y axes is reversed.

\end{document}
