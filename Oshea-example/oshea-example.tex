% !TEX TS-program = pdflatex
% !TEX encoding = UTF-8 Unicode

% This is a simple template for a LaTeX document using the "article" class.
% See "book", "report", "letter" for other types of document.

\documentclass[11pt]{article} % use larger type; default would be 10pt

\usepackage[utf8]{inputenc} % set input encoding (not needed with XeLaTeX)

%%% Examples of Article customizations
% These packages are optional, depending whether you want the features they provide.
% See the LaTeX Companion or other references for full information.

%%% PAGE DIMENSIONS
\usepackage{geometry} % to change the page dimensions
\geometry{a4paper} % or letterpaper (US) or a5paper or....
% \geometry{margin=2in} % for example, change the margins to 2 inches all round
% \geometry{landscape} % set up the page for landscape
%   read geometry.pdf for detailed page layout information

\usepackage{graphicx} % support the \includegraphics command and options

% \usepackage[parfill]{parskip} % Activate to begin paragraphs with an empty line rather than an indent

%%% PACKAGES
\usepackage{booktabs} % for much better looking tables
\usepackage{array} % for better arrays (eg matrices) in maths
 % \usepackage{paralist} % very flexible & customisable lists (eg. enumerate/itemize, etc.)
\usepackage{verbatim} % adds environment for commenting out blocks of text & for better verbatim
\usepackage{subfig} % make it possible to include more than one captioned figure/table in a single float
% These packages are all incorporated in the memoir class to one degree or another...

%%% HEADERS & FOOTERS
\usepackage{fancyhdr} % This should be set AFTER setting up the page geometry
\pagestyle{fancy} % options: empty , plain , fancy
\renewcommand{\headrulewidth}{0pt} % customise the layout...
\lhead{}\chead{}\rhead{}
\lfoot{}\cfoot{\thepage}\rfoot{}

%%% SECTION TITLE APPEARANCE
\usepackage{sectsty}
\allsectionsfont{\sffamily\mdseries\upshape} % (See the fntguide.pdf for font help)
% (This matches ConTeXt defaults)

%%% ToC (table of contents) APPEARANCE
\usepackage[nottoc,notlof,notlot]{tocbibind} % Put the bibliography in the ToC
\usepackage[titles,subfigure]{tocloft} % Alter the style of the Table of Contents
\renewcommand{\cftsecfont}{\rmfamily\mdseries\upshape}
\renewcommand{\cftsecpagefont}{\rmfamily\mdseries\upshape} % No bold!

%%% END Article customizations

%%% The "real" document content comes below...

\title{More Lessons from Quarantine}
\author{Jim OShea}
\date{2020-04-26} % Activate to display a given date or no date (if empty),
         % otherwise the current date is printed 

\begin{document}
\maketitle

\section{Introduction}

Welcome to the April 26 Northgate Bible Chapel Family Bible Hour.

We're doing this by video as we are in Covid-19 quarantine this morning I'm going to be sharing with
you a PowerPoint presentation so I'm just gonna get that set up right this second and there we go.

This message is entitled "More lessons from quarantine". I chose this because it's a continuation of a
variety of other messages that have been given in the last two months for the most of March and all of April.

We've been in this Covid-19 quarantine hopefully this is a once-in-a-lifetime type of situation.  With it,
you huge impact on every area of our lives.  Due to that a number of these messages have been focused
on dealing with this particular topic. I believe the Lord's been leading me in that direction as well, so I'd
like to build off of the March 28th message that Jeff Willard gave on kindness as one of the fruit of the Spirit. 
He was giving us a word of encouragement of how he has seen this assembly acting in very kind ways since
the beginning.

John Benson on April 5th had titled his message, "Lessons learned in isolation".  Now my situation is a little bit
different. I entitled this message, "More lessons in quarantine."

Much of my experience within the last couple of months has been going in the opposite direction because of the
work that I'm in. I've been pulled into a lot of hospitals in the last two months. That has added a lot of stress and
tiredness and anxiety to my life.

Last week Rich Greer's message on fixing our eyes on Jesus focused on the \textbf{solution} of the Lord Jesus Christ
as opposed to the \textbf{problems} that we've been dealing with. These are all the concepts that I think the Lord has
been working on in my heart as well. This has drawn me to this particular topic and where I am.

\section{John 15}

This is where the Lord has had me in the last few months is in John chapter 15 verse 11.  If you read it quickly
it's easy to \textbf{lose the impact}. I know I've read this many many times over the years and as I've done so
I've had to \textbf{force myself} to \textbf{slow down} and to \textbf{focus on it} and
 \textbf{to go word by word in this short sentence}.

If you're feeling much of the same way as I am hopefully, he'll do this with you. The the verse reads,

``These things I have spoken to you that my joy may remain in you and that your your joy may be full.''

Now it's this idea of joy that the Lord has really attracted me to. As I've been putting this message together,
the Lord has helped me to realize that a lot of what I've been struggling with in the last couple of months
regarding this global pandemic.

All the changes that have gone on the unknowns that have come into every area of our lives. He has
reminded me time and time again that He remains in control. As I turn my attention to this particular
verse, as I slow down and I focus on it, there are a few truths that I wanted to pull out that would help
us really focus in on what the Lord is saying.

The first is that it is the Lord Jesus Christ himself who was speaking to you and me. He says,

``I have spoken to you''

\section{The meaning of Joy}

As he does so, He is talking about Joy. The first point is "my joy", coming from His perspective, is God's joy.
The the second point is \textbf{our own joy}.

The word `joy` is used twice here. It is the same Greek word, Chara.  It means "to delight" or
"to give reason for gladness".

It is not necessarily a feeling. That is a \textbf{result}. Joy is something you choose. It is \textbf{not}
something that happens to you. \textbf{Joy} is the end result of a significant situation, or a perseverance,
endurance, or even present in a difficult situation -  like we are in right now. We \textbf{choose} to focus
on what we're going to talk more as the long game. We choose not being caught up in the short-term
problems - not being caught up in the situation, not being caught up the things that we cannot control.

Instead we \textbf{put on the full armor of God} and focus on on what we know is to be true. We have
faith in the plan of salvation that the Lord has given us.  We know how this is going to
end - we are told in the book of Revelation.

From that standpoint, so as we consider this idea of joy, notice that are two views of joy. One is owned
by God, and one is owned by you and I. Lastly these two Joy's are not guaranteed. The phrase,
\textbf{my joy may remain in you and that your joy may be full} are \textbf{conditional}. 
They may not be true that may not happen and there's a there's a very specific reasons for
that and the the verse starts off with the phrase these things it points to context this particular
sentence does not have any merit on its own unless you take it in context of John chapter 15
verses 1 through 10 and I would like to do that a little bit farther down in this message but
before we do I want to focus a little bit on the problem now.

\section{Joy under stress}

Rich did a great job last week of bringing out this particular truth and I'd like to build off that by
taking a little bit of talking a little bit about my journey in this corona-virus quarantine. I am tired.
I've been somewhat stressed, and certainly anxious as this unfolded.  At the beginning of March,
I was pulled more and more into hospitals that needed beds up and running. They needed their
system their IT system up and running as they were planning for an influx of covid19-patients.

- That meant long days.
- It meant projects that needed to be fast-tracked.
- It meant that what we had planned and the and how we were going to handle that all changed
very very quickly and a lot of this was outside of my control.

As I think about how it's impacted my family, how it's impacted our church family here at Northgate
Bible Chapel, how its impacted our extended families, it produces lots of stress, lots of concern
about our global economy. This has been both national and local. The shut down has an impact
that will be be felt for years to come.

That causes me a lot of concern. How am I supposed to be joyful in this uncertainty?  When you
look at what the media is talking about and all the negativity that goes along with it, we need to
come back to \textbf{Biblical Truth}.

The joy of Jesus didn't change. I did. I'm living these external circumstances. My emotions, my
short-term thinking, all of that is influencing me. I'm certainly too close to the problem and I'm
invested in a solution that will make me feel better. In a lot of ways my perspective here has me
feeling a lot like Peter. I'd like to look at \textbf{Matt 14:22-23}.

\section{Peter in Matt 14:22-23}

For the sake of time I'm going to ask you to stop the video and read this passage. Then come
back to the video. I don't want to go through this verse by verse. It's a very familiar passage.
There are a few things that I'd really like to pull about Peter here.

As I'm looking at this, Peter is in a really unique situation. The the Lord Jesus in \textbf{verse 29}
has called Peter to come. When Peter had come to him out of the boat, he walked on the water.
That is \textbf{incredible} by itself. Peter was walking on water. That is until he takes his eyes
off the Lord Jesus Christ he lets fear creep into his heart.  He allows the external circumstances
of life to distract him. The waves, the wind, and everything that goes along with it makes him
afraid.

Peter starts to sink. This is the best part of this story. Peter cries out, "Lord save me!". That is
exactly what the Lord did. Right around the corner, we have the Week of Prayer coming up.
I encourage all of us to attend.

I think it would be important to note here that this phrase,
\textbf{Lord save me} , that in of itself is a prayer. It's short. It's poignant.
The Lord responds! The best part is the Lord desires a relationship with us!

So as Peter is walking on the water, his (in v. 30) creeps into him he sinks starts sinking into
the choppy sea. He cries out to the Lord. In verse 31, immediately the Lord Jesus stretched
out his hands and catches Peter. Jesus says, "oh ye of little faith why did you doubt?" and
they got into the boat. Then the wind ceased. I may be projecting here, but I find it hard
to believe that once Peter was pulled out of these waves that he must have been delighted
and extremely focused on the reason for his gladness, going back to that definition of `Chora`
that we talked about previously.

The definition of the Greek word for joy the that Peter was having,  joy was delight in his
Lord and Savior, having that feeling of joy, that's going to come from the fact that he was
in a desperate situation but the Lord pulled him out of it.

Now as I examine my situation as I'm talking about the the fear and the anxiety and the
stress that I'm feeling. I find myself struggling with a conflict between fear and faith,
much of what like what Peter is doing here. There's a if I'm being critical of myself.

- I keep thinking of it. This is.
- These are basic things. This is Christianity 101.
- I know this stuff but why am I not living it?
- Why am I struggling with this?

That's a thought that's been playing me a little, and as I've gone through this particular
message in preparing for it I found a lot of peace because I think the Lord has shown
me exactly what I need to be focusing on in the days of the weeks to come.

Part of it here, as we find in the fruits of the Spirit in Galatians chapter five. About a month
ago Jeff Willard was teaching on the fruits of the Spirit and he focused specifically on kindness.
Again, encouraging us for the amount of kindness that he has seen the Saints at Northgate
show as an assembly.

Reacting to this sudden isolation now, I have to go back to this verse to focus on again
another basic truth, that the Holy Spirit is working in my life as a believer. When I am
being Spirit-led, these are the characteristics that I'm going to produce.

\section{Joy in Eph 6}

I will be producing joy but if I'm going to be honest with the other, as I was focusing on
this and while I was studying this, is because I'm not \textbf{experiencing joy} and
I've been forced to ask myself \textbf{why not}? I think this is where the Lord is taking
me. As a solution, I needed to be reminded that \textbf{joy is not a feeling, it's a decision}.
We are in a spiritual battle right now.

Eph 6:12 says,

``For we do not wrestle against flesh and blood, but against the rulers, against the authorities,
against the cosmic powers over this present darkness, against the spiritual forces of evil in the
heavenly places''

From a virus perspective we may be talking about life and death, but the the Lord Jesus Christ
conquered death, When he rose from the dead, from an economic standpoint all of these
resources that we have the money the jobs that are lost, they've all been given to us by the
Lord in the first place and he's asked us to be good stewards of them. So if they are his
good things to give us, they're also his good things to take away now. From a government
standpoint, the Lord has asked us to obey those who are the leaders that we are under,
unless they're asking us to do something unbiblical.

As we look at the broader context of Ephesians chapter 6, the Lord has given us the full
armor of God that we might be able to stand against the wiles of the devil. Now what's going
on here, is if we look at this differently, we're not talking about playing a short game here,
where we have these everyday comforts and they've been taken away from us. Now we're
struggling with that reality that we're talking about here is a long term fight. This is the
glory of God that we're talking about, and his eternal plan. Satan is attacking us he's
using us and I'm sure he is using a variety of different methods to do that. This one
of them (it sure seems that way with the coronavirus) and all the responses that are
going and the the isolation and the requests that we not meet together as we've been
called to do and everything that goes along with it.

This has been pulling up so many emotions but that is short term thinking as we as we
come to the idea of really what we're focused upon here what we're struggling with what
I'm struggling with is this idea that I can respond in fear or I can respond in faith and if
I'm gonna look at this as a spiritual battle I really only have one choice, right? I can only
I can only respond in faith as these external circumstances are pressing on us the the Holy
Spirit is able to produce joy in us when we allow him to do that.

\section{When I talk back to God}

The moment we take the wheel from him, the moment that we step in and say I got this,
that he's going to allow us and in our humaneness in our human nature, that's when the
fear kicks end that's where we seem to be taking in control saying:

\begin{itemize}
\item Lord I got this.
\item I know what's going on.
\item I know better than you.
\end{itemize}

That's not what the Lord is asking us to do, The thing about fear and faith, though they're
very closely aligned, \textbf{both require a belief that something hasn't happened yet}.
Core fear assumes a \textbf{negative result} and faith assumes a \textbf{positive result} .

If we look at this from a spiritual standpoint, that really gets us back to the our verse
\textbf{John 15:11}. We need to look at it in \textbf{context} . If we look at it in context we see that:

\begin{itemize}
\item  \textbf{my faith is in God} 
\item  \textbf{my faith is in the power of the Holy Spirit} 
\item  \textbf{my faith is in the fact that God has allowed these external circumstances to happen
to me for a very good reason} 
\item  \textbf{my faith is that when I read the book of Revelation I know how it's all going to end}.
I look at \textbf{Rev 21:22} , I know that God wins. My faith is that I get to expend a spend eternity
with him.
\item  \textbf{from God's perspective, not mine, but from God's perspective. I have nothing to fear}
\end{itemize}

That's true, yet I have a lifetime of experiences that shows me that \textbf{I can't do this} 

\begin{itemize}
\item  I don't have enough self-discipline
\item  I don't have enough grit
\item  I don't have Drive
\item  I don't have the strength to do it on my own
\end{itemize}

This is why we have to go back to John chapter 15 \textbf{in context} . So as we look at verses 1-11,
I'm going to ask you again to stop the video and to read this passage and then after you've read that
started again and we jump we'll jump in.

\section{Abiding in the Vine}

The first 5 verses here Jesus is telling us that he is the vine. Since I am a believer in the Lord Jesus Christ,
I'm one of the branches. So me, as a branch I connect to the vine that is my source of life, of everything,
and \textbf{as a result I produce fruit} .

I repeat, I produced that fruit \textbf{because I abide in him} and this word abide means:

- to stay, to remain, to continue, to conform.

I don't produce fruit \textbf{unless I am closely and intimately connected with the vine himself} and this
verse five here is one of my favorite verses because it's \textbf{a very direct statement} it says:

``I am the vine you are the branches he who abides in me and I in him bears much fruit for without me
you can do nothing''

`for without God I can do nothing`: that's a strong statement. Without God, I have no chance against
my fear, but with God my joy may remain in me. Jumping down to verse 11, see the connection here is
\textbf{that our joy is meant to and is designed to not come and go}.
\textbf{It is supposed to be constant it's supposed to remain just as you're supposed to remain
connected to the the vine you as the branch}.

How is this all? We remain together. How does this all work? Well it comes to this idea of abiding in
God. Abiding in the vine each and every day. Experiencing life with Jesus. When I do this, when I
experience life on an everyday sort of basis, my fear has no chance.

\section{On the road to Emmaus}

So, what I want to do here is give you an example. So this example that I'm going to take a look a
here, has two disciples who are struggling with their external circumstances, much like we are right
now. It's not a one-to-one comparison, but I think for the purpose of this, you'll see where I'm going now.
They had an experience with Jesus that resulted in a lasting joy, and that is on the road to Emmaus.
Stop the video here and read \textbf{Luke 24:13-3} and then we'll pick up after you've done.

\begin{quote}
\textsuperscript{13}That very day two of them were going to a village named Emmaus, about seven
miles from Jerusalem, \textsuperscript{14}and they were talking with each other about all these
things that had happened. \textsuperscript{15} While they were talking and discussing together,
Jesus himself drew near and went with them. \textsuperscript{16} But their eyes were kept from
recognizing him. \textsuperscript{17} And he said to them, "What is this conversation that you are
holding with each other as you walk?" And they stood still, looking sad. \textsuperscript{18} Then
one of them, named Cleopas, answered him, "Are you the only visitor to Jerusalem who does not
know the things that have happened there in these days?" \textsuperscript{19} And he said to them,
"What things?" And they said to him, "Concerning Jesus of Nazareth, a man who was a prophet
mighty in deed and word before God and all the people, \textsuperscript{20} and how our chief
priests and rulers delivered him up to be condemned to death, and crucified him.
\textsuperscript{21} But we had hoped that he was the one to redeem Israel. Yes, and besides
all this, it is now the third day since these things happened. \textsuperscript{22} Moreover, some
women of our company amazed us. They were at the tomb early in the morning,
\textsuperscript{23} and when they did not find his body, they came back saying that they had
even seen a vision of angels, who said that he was alive. \textsuperscript{24} Some of those who
were with us went to the tomb and found it just as the women had said, but him they did not see."
\textsuperscript{25} And he said to them, "O foolish ones, and slow of heart to believe all that the
prophets have spoken! \textsuperscript{26} Was it not necessary that the Christ should suffer these
things and enter into his glory?" \textsuperscript{27} And beginning with Moses and all the
Prophets, he interpreted to them in all the Scriptures the things concerning himself.

\textsuperscript{28} So they drew near to the village to which they were going. He acted as if he were
going farther, \textsuperscript{29} but they urged him strongly, saying, "Stay with us, for it is toward
evening and the day is now far spent." So he went in to stay with them. \textsuperscript{30} When he
was at table with them, he took the bread and blessed and broke it and gave it to them. 
\textsuperscript{31} And their eyes were opened, and they recognized him. And he vanished
from their sight. \textsuperscript{32} They said to each other, "Did not our hearts burn within
us while he talked to us on the road, while he opened to us the Scriptures?" 
\end{quote}

So the story here is the the two disciples. It's after the Lord has been crucified and it's during his
resurrection from from the dead and now he's appearing to specific disciples and as they're walking
to Emmaus. You see in verses 13-16, but they're talking with one another. In verse 17, the Lord
Jesus Christ asked them what kind of conversation is this, that you have one with another as you
walk, and are sad. So they're talking about all the bad things that have been happening and they're
really emotionally invested here. You know you see the word sad and the way that they're responding
to all of this very similar to what we're dealing with right now. Then we see in verses 18-19 they
continue on to conversation and the Lord asks them "what things?", so then they give a bit more
of the perspective that they're coming from. So they said to him these things concerning Jesus of
Nazareth, He was a prophet mighty, in deed and word before God, and all the people, and then
they go on to their summary of the situation and how it manifested himself. Then in verse 25
he said to them:

\begin{quote}
O foolish ones and slow of heart to believe
\end{quote}

Again we're talking about faith here and

\begin{quote}
all that the prophets have spoken ought not the Christ to have suffered these things?
\end{quote}

Again we're coming to the idea of suffering right now

\begin{quote}
and to enter into his glory and beginning with Moses and all the prophets he expounded to
them in the Scriptures the things concerning himself
\end{quote}

It goes on here in verse 28. I'm going into verse 29. Note the word `abide`.

\begin{quote}
Then they drew near to the village where they were going and he indicated that as he would
have gone further but they constrained him saying abide with us
\end{quote}

Going back to that John 15 idea...

\begin{quote}
stay with us
\end{quote}

The disciples wanted to be connected with the Savior. The branch to the Vine. We can do
nothing aside from Christ who strengthens us.

But he goes on to say:

\begin{quote}
abide with us for his toward evening and the day is far spent and he went in to stay with them now it
came to pass as they sat at the table with them he took bread blessed and broke and gave it to them
then their eyes were opened and they knew him and he vanished from their sight.
\end{quote}

So this ties back very nicely to the April fourth message that her brother John Benson gave on
\textbf{Acts 2:42} and the the lessons of isolation that he was talking about.

So we're seeing a lot of the the steps and the the tools that God has given us and from from Rich
and from Jeff and from John as well.

\section{How to view trials}

So what I'm hoping to do here is give you a bit of the the framework around that of how we are
supposed to be looking at during these times that we are in right now.

Because verse 32 ends and this is where I was kind of coming to:

\begin{quote}
and they said to one another did not our hearts burn within us while he talked with us on the road
while he opened the Scriptures with us.
\end{quote}

That certainly sounds like joy to me. As they were thinking about the delights and the reason for gladness
it went back to the person the experience with Jesus. Now this was an everyday experience these two had
gone on a walk together they're going to Emmaus but they added by adding a meaningful experience with
Jesus. It turned into pure joy. The fear, the sadness, and the anxiety they all melted away. That's the
experience that we're looking for as we go through these struggles of life as we're dealing with so much
uncertainty right now. As I started this by by confessing to you that I've been struggling with joy lately,
in essence what I was saying, and the point I want to get to here, as we deal with our problems we're
in a lot of ways saying we just don't like the current challenges that were in.

I want an easier life. That's really not realistic. If we come to some chapter 23 and I want to focus on
verse 4 but I would encourage you to stop the video here and read the entire Psalm, but picking up
in verse 4 we were promised,

\begin{quote}
Yea though I walk through the valley of the shadow of death I will fear no evil for you are with me your
rod and your staff they comfort me.
\end{quote}

We are not being promised here a trouble-free life. The path is going through the Valley of the Shadow of Death,
but as a believer I am promised that God is going to go with me every step of the way as I go through tough
times. He's gonna comfort me along the way these hardships that were going through. They require endurance
and that's where Rich had us last week into Hebrews chapter 12. He was looking at verses 1-2 now. I normally
don't read the the NLT version, but I really liked what it was saying here, so I'm sharing it with you.

\begin{quote}
Therefore since we are surrounded by such a huge crowd of witnesses to the life of faith, let us strip off
every weight that slows us down, especially the sin that so easily trips us up, and let us run with endurance
the race of God has set before us. We do this by keeping our eyes on Jesus, the champion who initiates
and perfects our faith. Because of the joy awaiting Him, He endured the cross, disregarding its shame,
 Now He's seated in the place of honor besides God's throne.
\end{quote}

\section{Keeping our eyes on Jesus}

So to build off of what Rich was talking about last week, the problem of taking our eyes off Jesus, is that

- when we are solution focused,    
- when we are faith based when we see the world from God's perspective     
- and when we take our eyes off Jesus we became problem focused we became become short-sighted.

Recently, we get caught up in the emotion we can't see beyond ourselves. But here we have the example
of God - God himself he endured the cross he disregarded the shame why because of the joy that was
waiting before him.

This is what brings what we're going through right now:

- meeting our delight
- a reason for gladness
- is actually as weak, to quote the end of verse two
- Christ is now seated in place of honor beside God' throne
- He is our joy in this spiritual battle.

We're called to play this one long game:

- just as Jesus was, because of the joy awaiting him,
- because of God's honor
- because of God's plan
- because how we know this all is all going to end

What we're going through in these trying times right now, it is pales in comparison to what the the
blessings that the Lord has for us in the in the age to come. So when we have this focus, when we
look at it this, way when we are living by faith not by fear, we are then a living witness to the power of God.

\section{Truth from Hebrews 12}

So verse one of Hebrews 12 says,

\begin{quote}
Therefore, since we are surrounded by such a huge crowd crowd of witnesses to the life of faith
\end{quote}

That's in essence what we are we're a witness to the - life of faith - so when we have this positive
attitude going through these difficult times right now, and other people see, it that's when they
start asking us questions like:

\begin{quote}
Why are you handling this life self-isolation so well when everyone else isn't well?
\end{quote}

Our response should be:

\begin{quote}
Let me tell you about the Gospel. Let me tell you about the the fruit of the Spirit. Let me tell you
about how I know something better is coming.
\end{quote}

That's when you're going \textbf{to be ready in season} to share the Gospel in a time of uncertainty.

\section{Closing thoughts}

To close these thoughts here, our time of trouble-free fellowship with God is actually in the future.
That easy life that we so want today, that's Revelation 21, That's coming.  We're gonna have
no tears, no death, no sorrow, no pain, the former things will have passed away.

In the meantime, you're fixing your eyes on Jesus.  Remember that we have a choice, and our
choice is to live by faith, not by fear, and now along those lines as we live this life 
\textbf{attitude is everything} !

We're going through some trying times right now but God His Word has given us some
very practical ways that we can live for Him.  We can show the world his glory. We can be
this witness that he has called us to do and brother Jeff talked about that on March 28th.
A little bit where he talked about us showing kindness. Now here's some more.
Now this is not an exhaustive list, but it's a good place to start. 

- Remember that the Lord is asking us to believe unto salvation - John 3:16. 
- Remember that He desires us to repent of sin these are things that we can continue to do now.
He desires that we should pray regularly and that we should pray without ceasing and then being
completely dependent on God without giving up. We keep praying. We keep praying. And we
keep praying. We should be practicing gratitude (1st Thess 5:16) 
- We should serve each other (Mark 10:42-45)
- We've been created as unique individuals, in God's image, (Eph 2:10) so just be yourself.

And that's saying that more for me than anybody else. I just gotta be me because this is the way
the Lord made me and in 1 Pet 3:15 and we keep things right with God we're so in love with him we
can't help but talk about him not because I have to but because I want to. If I'm putting the eggs all
into perspective here (Phil 4:6-7) just don't worry. The Lord's got a plan let tomorrow to care of itself.
Take it one day at a time (Matt 6:33-34) and be a person of action by doing good things (James 1:22-25)

Again, this is not an exhaustive list but these are really very real very practical ways that we're going to
be able to live by faith, here and now, everyday meeting Jesus where we're at, and when we get to
those difficult situations turn to God. Remember his perspective: \textbf{Don't live by fear, live by faith, love}.
I look forward to seeing you soon. I'm praying for you. Please pray for me and as this ends. I look forward
to that renewed fellowship that will be super-sweet with each other. Have a good day. bye.

\end{document}
