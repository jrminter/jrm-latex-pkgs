% !TEX TS-program = xelatex
% !TEX encoding = UTF-8 Unicode

% A simple template for a LaTeX document using the "article" class.

\documentclass[12pt]{article} % use larger type; default would be 10pt

%\usepackage[utf8]{inputenc} % set input encoding (not needed with XeLaTeX)

%%% Examples of Article customizations
% These packages are optional, depending whether you want the features they provide.
% See the LaTeX Companion or other references for full information.

%%% PAGE DIMENSIONS
\usepackage{geometry} % to change the page dimensions
\geometry{letterpaper} % or letterpaper (US) or a5paper or....
\geometry{margin=1.0in} % for example, change the margins to 2 inches all round
% \geometry{landscape} % set up the page for landscape
%   read geometry.pdf for detailed page layout information

\usepackage{graphicx} % support the \includegraphics command and options
\usepackage{xcolor}
\definecolor{kodak-red}{RGB}{204,0,0}
\definecolor{kodak-yellow}{RGB}{255,204,0}

%\usepackage[parfill]{parskip} % Activate to begin paragraphs with an empty line rather than an indent

%%% PACKAGES
\usepackage{booktabs} % for much better looking tables
\usepackage{array} % for better arrays (eg matrices) in maths
\usepackage{paralist} % very flexible & customisable lists (eg. enumerate/itemize, etc.)
\usepackage{verbatim} % adds environment for commenting out blocks of text & for better verbatim
\usepackage{subfig} % make it possible to include more than one captioned figure/table in a single float
\usepackage{parskip}
% These packages are all incorporated in the memoir class to one degree or another...

%%% HEADERS & FOOTERS
\usepackage{fancyhdr} % This should be set AFTER setting up the page geometry
\pagestyle{fancy} % options: empty , plain , fancy
\renewcommand{\headrulewidth}{0pt} % customise the layout...
\lhead{}\chead{}\rhead{}
\lfoot{}\cfoot{\thepage}\rfoot{}

%%% SECTION TITLE APPEARANCE
\usepackage{sectsty}
\allsectionsfont{\sffamily\mdseries\upshape} % (See the fntguide.pdf for font help)
% (This matches ConTeXt defaults)

%%% ToC (table of contents) APPEARANCE
\usepackage[nottoc,notlof,notlot]{tocbibind} % Put the bibliography in the ToC
\usepackage[titles,subfigure]{tocloft} % Alter the style of the Table of Contents
\renewcommand{\cftsecfont}{\rmfamily\mdseries\upshape}
\renewcommand{\cftsecpagefont}{\rmfamily\mdseries\upshape} % No bold!

% define degree symbol
\newcommand{\degree}{\ensuremath{^{\circ}}}

% define commands for subscripts & superscripts to use in text mode
\newcommand{\superscript}[1]{\ensuremath{^{\textrm{#1}}}}
\newcommand{\subscript}[1]{\ensuremath{_{\textrm{#1}}}}

%%% END Article customizations
\parskip=12pt
%%% The "real" document content comes below...

\title{Grow and Serve Christ in the New Year}
\author{John Minter}
\date{2020-01-05} % Activate to display a given date or no date (if empty)

\begin{document}
\maketitle

\section{Introduction}

This morning we all have something in common:

\begin{itemize}
	\item We each have a new year.
	\item We each will make choices, both good and bad.
	\item For those of us who profess faith in Jesus Christ, our choices \emph{reflect on Him}..
\end{itemize}

Our passage for the morning is \textbf{2 Tim 1:3-2:26}

Our question - and title for this morning's message is:
\textbf{How will I serve Christ in 2020?}

\section{Paul's relationship with Timothy}

Consider Paul's relationship with Timothy.

Paul reminds Timothy - and all of us - of the nature of that relationship in
verses \textbf{3--5}.

\begin{quote}
\superscript{3}I thank God whom I serve, as did my ancestors, with a clear
conscience, as I remember you constantly in my prayers night and day.
\superscript{4}As I remember your tears, I long to see you, that I may be
filled with joy. \superscript{5}I am reminded of your sincere faith, a faith
that dwelt first in your grandmother Lois and your mother Eunice and now,
I am sure, dwells in you as well.
\end{quote}

Paul knew Timothy and his family well.

In \textbf{Acts 16}, we learn that Timothy accompanied Paul on his second
missionary journey.

We will see later in our passage, that Timothy's mother and grandmother had
come to Christ when Paul preached in their home town of Lystra.

The book of Acts portrays Timothy as
a trustworthy assistant that Paul could send as his representative.

That trust gave Paul confidence that Timothy could handle
difficult spiritual situations.

Paul wanted Timothy to continue to grow in Christ. (\textbf{long pause})

He \emph{knew} Timothy would
face difficult times, as we are are likely to face difficult times.

Paul lays the ground work for this in verses \textbf{6-7}.

Note the imagery Paul uses.

\begin{quote}
\superscript{6}For this reason I remind you to fan into flame the gift of God,
which is in you through the laying on of my hands, \superscript{7}for God gave
us a spirit \emph{not of fear but of power and love and self control}.
\end{quote}

The Spirit had used Paul to equip Timothy for ministry. 

Paul paints the picture of starting a fire.

The hymn writer gives us this:

\begin{flushleft} 
\textbf{
It only takes a spark\linebreak
To get a fire going\linebreak
And soon all those around\linebreak
Can warm up in its glowing\linebreak
That's how it is with God's love\linebreak
Once you've experienced it\linebreak
You spread His love to ev'ryone\linebreak
You want to pass it on  \linebreak
}
\end{flushleft}     

For Timothy  -- \textbf{and us} - -  to be a useful servant of Christ, we need to fan
the flames and let the Holy Spirit develop our spiritual gifts.

The Spirit uses many means to equip each of us.

\begin{enumerate}

\item He uses the Scriptures. As we read and meditate on them and discuss
them with one another we build each other up.

\textbf{  Prov 27:17} teaches us:

\begin{quote}
\superscript{17}Iron sharpens iron, and one man sharpens another
\end{quote}  . 

\item As we go through life together in Christian fellowship,
we serve and strengthen one another.

\item As we face difficult situations, the Spirit of God \textbf{comforts}
us and {equips us to face them}. 


\end{enumerate}

Serving Christ will come at a price.   (\textbf{long pause})

In verses \textbf{8-12} we see that Paul gets \textbf{really excited}!

You may ask,

\begin{quote}
How do we \textbf{know} Paul is excited?"
\end{quote}

We know because \textbf{vs 8-12}is a run-on sentence in \textbf{both}
Greek and English.

Paul is on a roll and doesn't want to stop for a breath.     
(spaces for my eyes...)

I'm going to break this into smaller chunks to better help us understand
Paul's advice.

Note how Paul begins in v. 8:

\begin{quote}
\superscript{8} Therefore do not be ashamed of the testimony about our
Lord, nor of me his prisoner, but
\textbf{share in suffering for the gospel by the power of God}.
\end{quote}

Paul warns - do not be ashamed when people reject your gospel message.

Indeed, Paul wants Timothy - \textbf{and us} - to \textbf{expect opposition}.

Nobody likes facing the fact that \textbf{they}
- along with the rest of the human race -
are \textbf{sinners} who \textbf{deserve the wrath} of God.

Most are \textbf{offended} and stop listening at this point.

Only the Spirit of God can open hearts and minds!

Paul goes on to explain in \textbf{vs 9}, that this is part of God's plan.

\begin{quote}
\superscript{9}who saved us and \textbf{called us to a holy calling},
\textbf{not because of our works \emph{but}  because of \emph{his own purpose and grace}},
which \textbf{he gave us in Christ Jesus before the ages began}.
\end{quote}

Christ's death on the cross \textbf{in the place of all who receive Him}
is \textbf{the means} by which God \textbf{forgives} us.

Because the sinless Christ died in the place of each person who receives
and believes His gospel, God the Father can declare that the person's
sin was removed by the sinless Christ paying that person's debt by His
by His death.

This is the good news of the Gospel! That is good news that needs to reach
people from all the earth. 

Paul explains how this works in V. 11.

Only the Spirit of God can open hearts and minds.

Paul goes on to explain in vs 9, that this is part of God's plan

\begin{quote}
\superscript{9} who saved us and \textbf{called us to a holy calling},
\textbf{not because} of our works
\textbf{but because of his own purpose and grace},
which \textbf{he gave us in Christ Jesus before the ages began}.
\end{quote}

Paul wants Timothy - and us - to understand that all of this is God's
plan for the gospel.

Paul continues to explain:

\begin{quote}
\superscript{10}and which now has been 
\textbf{manifested} through the \textbf{appearing} of our Savior Christ Jesus,
who \textbf{abolished death and brought life and immortality to light through the gospel}.
\end{quote}

Christ's death on the cross \textbf{in the place of all who receive Him}  is
the \textbf{means} by which God forgives us.

Because the sinless Christ died in the place of each person who receives
and believes His gospel, God the Father can declare that the person's
sin was removed by the sinless Christ paying that person's debt by His
by His death.

This is the good news of the Gospel!

That is good news that needs to reach people from all the earth.

Paul explains how this works in v. 11:

\begin{quote}
\superscript{11}for which I was appointed a preacher and apostle
and teacher, \superscript{12}which is why I suffer as I do. But I am not
ashamed, for \textbf{I know whom I have believed}, and
\textbf{I am convinced that he is able to guard}
until that day what has been entrusted to me.
\end{quote}

Preaching the good news is a privilege.

That privilege comes with persecution.

Despite persecution, God is still in control and His messengers will
both be vindicated - as Christ was vindicated by His resurrection from
the dead.

Paul describes to Timothy how he had come to view his own ministry.

Paul wanted to strengthen Timothy \textbf{for} ministry.

Note the advice Paul give in verses 13 and 14:

What a gift God has given us - \textbf{the Gospel Message}!


In verses 15-17, Paul warns Timothy that
\textbf{opposition accompanies ministry}.

\begin{quote}
\superscript{15}You are aware that all who are in Asia turned away
from me, among whom are Phygelus and Hermogenes.
\superscript{16}May the Lord grant mercy to the household
of Onesiphorus, for he often refreshed me and
\textbf{was not ashamed of my chains},
\superscript{17}but when he arrived in Rome he searched for me earnestly
and found me — \superscript{18} may the Lord grant him to find mercy
from the Lord on that day! — and you well know all the service he
rendered at Ephesus.
\end{quote}

Think for a moment:

\begin{itemize}
\item What conclusions do you think Paul wanted Timothy to draw? (pause)
\item What conclusion does the Spirit of God want \textbf{us} to draw? (pause)
\end{itemize}

I think Paul wants all who read this to share the Gospel \textbf{despite}
opposition. 

\section{A Good Soldier for Christ Jesus}

Paul will explain this to Timothy - and us - using the analogy of a
\textbf{good soldier}.

My son, Tim, is a Major in the US Army. He serves as a Lawyer. He explains to
his clients - high ranking officers - that his job is to make sure that they
do not have to explain illegal decisions to congressional oversight.

You can imagine that he encounters resistance and opposition at times.

That comes with the job. Paul will emphasize the good soldier's actions
and motives in \textbf{chapter 2 verses 1-4}.

\subsection{Paul's analogy of a good soldier}


\begin{quote}
\superscript{2:1}You then, my child, be strengthened by the grace that
is in Christ Jesus, \superscript{2}and what you have heard from me
\textbf{in the presence of many witnesses entrust to faithful men, who
will be able to teach others also. \superscript{3}Share in suffering
as a good soldier of Christ Jesus.} \superscript{4}No soldier gets
entangled in civilian pursuits, since his aim is to please the one who
enlisted him.
\end{quote}

\textbf{To summarize}

The Christian's job is to \textbf{faithfully proclaim the good news of Christ}.
If he is going to suffer and die - it will be \textbf{in the right battle}
on \textbf{the right hill}, \textbf{bringing glory to Christ}.

\subsection{Paul's analogy of those who work and train hard}

Paul wants Timothy to be effective. Paul uses a second analogy.
\textbf{An Olympic athlete}. Let's see how Paul explains this in
verse s {5-7}:

\begin{quote}
\superscript{5}An athlete is not crowned unless he competes according to the rules.
\superscript{6}It is the hard-working farmer who ought to have the first share of the crops.
\superscript{7}Think over what I say, for the Lord will give you understanding in everything.
\end{quote}

Think of a successful athlete. Let me suggest the golfer, Tiger Woods.

Here is how Tiger Woods described his training:

\begin{quote}
Well, I used to get up in the morning, run four miles, Woods said.
Then I’d go to the gym, do my lift. Then I’d hit balls for two to three hours.
I’d go play, come back, work on my short game. I’d go run another four more
miles, and then if anyone wanted to play basketball or tennis, I would go play
basketball or tennis. That was a daily routine.
\end{quote}

How would our ministry look if we put in that level of focus?

When we see an effective athlete or one who faithfully and effectively
ministers the Word, we should think of an iceberg.

What we see is the little bit that shows above the water.

What we \textbf{don't see} is all the work of preparation done in private
by both the speaker and the Holy Spirit. 

All of this works together to change hearts and lives for Christ. 

Success requires \textbf{both persistence and preparation}!

\subsection{The centrality of Christ}

Paul want to make sure we focus our attention correctly so we
\emph{hit the target} \textbf{in spite of opposition or problems}.

Note how Paul describes this in \textbf{verses 8-9}:

\begin{quote}
\superscript{8}Remember Jesus Christ, risen from the dead, the offspring
of David, as preached in \textbf{my gospel},
\superscript{9}for which \textbf{I am suffering, bound with chains as a criminal}.
\textbf{\emph{But the word of God is not bound!}}
\end{quote}

Paul \textbf{did not let bad circumstances keep him down}.

\subsection{Paul's personal confession }

Paul had a personal creed - think of it as a description of his life's 
mission. He explains in verse 10:
`
Paul will follow up with what appears to be an early Christian "creed" -
a summary of beliefs that could be easily memorized.


\begin{quote}
\superscript{10}Therefore \textbf{I endure everything for the sake of the elect}
 that they also may obtain the salvation that is in Christ Jesus with eternal glory.
\end{quote}

Note how Paul kept his eye on the prize. \textbf{May we be like him}! 



Paul will follow up with what appears to be an early Christian "creed" -
a summary of beliefs that could be easily memorized.

Remember, \textbf{very few people had their own copy of the scriptures at the time}.

There was no \textbf{Google}! 

\subsection{An early creed}

Let's look at verses \textbf{11-13}

\begin{quote}

\superscript{11}The saying is trustworthy, for:
If we have died with him, we will also live with him;     
\superscript{12}if we endure, we will also reign with him;     
if we deny him, he also will deny us;     
\superscript{13}if we are faithless, he remains faithful
—for he cannot deny himself.
\end{quote}

These are great verses to ponder...



What a way to describe a Lord who loves us knowing that at times
we fail.

As long as we continue to \textbf{love Him} and \textbf{believe in Him}
during good and bad times, \textbf{Christ will do what He promised}.

\subsection{A Worker Approved by God}

Paul's goal is that Timothy - and all of us who read this - 
be \textbf{life-long effective servants of God}.

That does not \textbf{"just happen"}.

It requires each of us to yield to the Spirit of God who seeks
to conform us to the character of Christ and equip us with
spiritual gifts required to complete His work through us.

Beginning in v. 14, Paul explains how that happens... Which I will
convert to a bullet list for an American audience.




\begin{itemize}
\item  Don't quarrel

\superscript{1}Remind them of these things, and
charge them before God not to quarrel about words, which does no good,
but only ruins the hearers.

\item Do your best to serve Christ.

\superscript{15}Do your best to present yourself to God as one approved,
a worker who has no need to be ashamed, rightly handling the word of truth.
Handle God's word with care!

\item Ignore unhelpful controversies.

\superscript{16}But avoid irreverent babble, for it will lead people into more
and more ungodliness, \superscript{17}and their talk will spread like gangrene.
Among them are Hymenaeus and Philetus, \superscript{18}who have swerved from the
truth, saying that the resurrection has already happened. They are upsetting the
faith of some.

A meme from the 1990's - before memes were a thing - is helpful here
  
\textbf{Don't feed the trolls}

With the advent of the Internet and social media came onslaughts of drivel
from bored, contentious, self important people who seem to thrive on
controversy.
  
They love to stir the pot.
  
I encountered my first troll in the 1990's when "Science Newsgroups" were
a popular on the Internet. They were helpful for those of us with limited
travel budgets and couldn't go to the cool conferences and workshops.
  
A student at Dartmouth was bored and contentious and loved spamming the
newsgroups. He used the ID `Archimedes Plutonium`. He spammed most of the
science newsgroups with convoluted screed proclaiming that the Plutonium
atom was God.
  
He \textbf{knew} it was off topic and \textbf{loved to stir the pot}.
  
His spam made the newsgroups useless.
  
\textbf{\emph{Don't feed the trolls!}}

\item{\textbf{Relax!}, God is sovereign and in control. This is good news!}

Let's see how Paul explains this in verse 19:
  
\superscript{19}But God’s \textbf{firm foundation stands}, bearing this seal:
The Lord knows those who are his, (\textbf{Korah in Num 16:5?})
and, Let everyone who names the name of the Lord depart from iniquity
\textbf{Isa 26:13}.

When we are overwhelmed, we need to pause and remind ourselves that God is
sovereign and is in control right now in the situation we are facing. 
  
He lets us relax and get beck to work and be productive for Christ.

\item Be content for God to use you in whatever roll He pleases.

This is much easier said than done. Too often we are like James and
John who picked the \textbf{worst possible moment} to be \textbf{selfish}.

Christ had just dropped a bomb shell in \textbf{Mark 10:33-34}:

\superscript{33}saying, See, we are going up to Jerusalem, and the Son of
Man will be delivered over to the chief priests and the scribes, and they
will condemn him to death and deliver him over to the Gentiles.
\superscript{34}And they will mock him and spit on him, and flog him and
kill him. And after three days he will rise.

So \textbf{what} did James and John ask Christ in \textbf{that context}?

Let's look at \textbf{Mark 10:35–41}.

This is \textbf{the passage} that should convince us to
\textbf{think before we speak}.

\superscript{35}And James and John, the sons of Zebedee, came up to him and
said to him:    
Teacher, we want you to do for us whatever we ask of you.
\superscript{36}And he said to them, What do you want me to do for you?
\superscript{37}And they said to him, Grant us to sit, one at your right hand
and one at your left, in your glory.

\superscript{38}Jesus said to them, You do not know what you are asking.
Are you able to drink the cup that I drink, or to be baptized with the baptism
with which I am baptized?

Jesus is \textbf{strongly recommending} they reconsider.

James and John show that they haven't thought this through... (pause)    
and they \textbf{double down}. This will not end well...:

\superscript{39}And they said to him, We are able.

Jesus corrects them:

And Jesus said to them, The cup that I drink you will drink, and with the
baptism with which I am baptized, you will be baptized,

Here, Jesus explains that they will suffer as His disciples. He will also
indicate in verse 40 that the Father will choose who sits where...

\superscript{40}but to sit at my right hand or at my left is not mine to grant,
but it is for those for whom it has been prepared.

My parents used to describe such an exchange as taking a person down a peg or two.

Recall that the other disciples heard this discussion that James and John had with Jesus.

The other disciples \textbf{were not pleased}.

\superscript{41}And when the ten heard it, they began to be \textbf{indignant}
at James and John.

Jesus' point is that we should be \textbf{content} to serve Him in
\textbf{any manner that pleases Him}.

When you \textbf{start at the bottom}, the \textbf{only place to go} is
\textbf{up}!

In \textbf{2 Tim 2:20}, Paul will explain it like this:

\superscript{20}Now in a great house there are not only vessels of gold
and silver but also of wood and clay, some for honorable use, some for
dishonorable.
  
Paul argues that we are all different and God has tasks for all of us.
Our choices impact our usefulness to God. Paul explains:
  
\superscript{21}Therefore, if anyone cleanses himself from what is dishonorable,
he will be a vessel for honorable use, set apart as holy, useful to the
master of the house, ready for every good work.
  
Paul now explains what this looks like in practice.

He starts with a positive admonition:
  
\superscript{22} So \textbf{flee} youthful passions and \textbf{pursue}
righteousness, faith, love, and peace, along with those who call on the
Lord from a pure heart.
  
He follows up with a negative admonition:
  
\superscript{23}Have \textbf{nothing to do with foolish, ignorant controversies;}

Paul explains:
  
\textbf{you know that they breed quarrels}.
  
In other words, Paul says, \textbf{you know better than to do this}.
  
Paul wants to be \textbf{certain} we get the point, so he makes a contrast
in \textbf{vss. 24-26}. 
  
\superscript{24} And the Lord’s servant must not be quarrelsome but kind to
everyone, able to teach, patiently enduring evil, \superscript{25}correcting
his opponents with gentleness. God may perhaps grant them repentance leading
to a knowledge of the truth, \superscript{26}and they may come to their senses
and escape from the snare
of the devil, after being captured by him to do his will.
  
This is Paul's great summary statement.

Art Taylor calls it \textbf{getting a listening ear}.

Paul - or someone close to him - put it this way in \textbf{Heb 4:11–13},
where he uses the term \textbf{that rest} to describe the peace with God
that comes from receiving the Gospel - the good news - of Jesus Christ.



\textbf{Heb 4:11–13}

\superscript{11} Let us therefore strive to enter that rest, so that no
one may fall by the same sort of disobedience.

Paul is referring to the Israelites who refused to enter the Promise Land

\superscript{12} For the word of God is living and active, sharper than
any two-edged sword, piercing to the division of soul and of spirit, of
joints and of marrow, and discerning the thoughts and intentions of the
heart.
\superscript{13}And no creature is hidden from his sight, but all are naked
and exposed to the eyes of him to whom we must give account. 

When we study the Scriptures and listen, we see our sin. God planned it this way.
God expects us to confess our sin and repent.

When we obey, the Holy Spirit transforms our character to make us more
Christ-like.

This changes makes us more faithful representatives for Christ and pleases Him.']


\end{itemize}

The question for each of us this morning is

\textbf{How will I serve Christ this year?}

1. If you have never heard the good news of the gospel - That is that Christ
death on the cross paid the penalty for the sin of all who come to Him, please
accept His gift. If you have questions about the gospel, ask one of us. We would
love to answer your questions.

2. For those of us who have received Christ, we need to ask ourselves:
How will I serve Him and bring glory to His name this year. Paul's
advice to Timothy is great counsel!

Please ponder the relevant question for a moment... (long pause)

Let's close with prayer. (pray)



% More text.

\end{document}
